%!TEX root = ../dokumentation.tex

\chapter*{\langabkverz}
%nur verwendete Akronyme werden letztlich im Dokument angezeigt
\begin{acronym}[YTMMM]
\setlength{\itemsep}{-\parsep}

\acro{AGPL}{Affero GNU General Public License}
\acro{PWM}{Pulsweitenmodulation}
\acro{CAD}{}
\acro{CERN}{Europäische Organisation für Kernforschung}
\acro{SPICE}{Simulation Program with Integrated Circuit Emphasis}
\acro{TWI}{Two Wire Interface: Serieller Multimaster-Bus über zwei Leitungen, auch I\textsuperscript{2}C genannt}
\acro{GUI}{Graphical User Interface, Grafische Benutzerschnittstelle}
\acro{LED}{Licht-emittierende Diode}
\acro{GPIO}{General-purpose input/output: Programmierbare digitale Ein- \& Ausgangspins}
\acro{HDMI}{High-Definition Multimedia Interface, digitale Schnittstelle z.B. für Bildschirme}
\acro{ISP}{In-System-Programming: Schnittstelle, mit der Mikrocontroller im eingebauten Zustand programmiert werden können}

\end{acronym}
