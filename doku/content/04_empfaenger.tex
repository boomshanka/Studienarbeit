%!TEX root = ../dokumentation.tex

% Kapitel über Empfänger-Schaltung, Signalauswertung
% Idee, Umsetzung etc


\chapter{Empfänger}
Der Empfänger ist einer der wichtigsten Bausteine bei der Abstandsmessung mit Ultraschall. Nur durch eine zuverlässige Signalverarbeitung kann der Zeitpunkt exakt bestimmt werden, an dem das Signal eingetroffen ist. Im Prinzip gibt es zwei Möglichkeiten die Signalauswertung durchzuführen:
\begin{description} % FIXME: Vielleicht in eigenen sections sinnvoll
	\item[Signalauswertung mit AD-Wandler:] Das empfangene Signal wird vorverstärkt und ständig von einem AD-Wandler ausgelesen. Die Auswertung kann über einfache Schwellwerte bis hin zu komplexen Filterfunktionen, z.B. mithilfe einer Fourier-Transformation, beliebig aufwendig gestaltet werden. Eine Signalvorverarbeitung ist auch denkbar, z.B. Gleichrichten und Glätten, wenn nur die Auslenkung betrachtet wird. Software kann sich auf die Umgebungsbedingungen anpassen, ist jedoch ungenauer und langsamer als die entsprechende Hardwarelösung. Eine genaue Frequenzauswertung erfordert eine hohe Abtastrate und komplexe Software.
	\item[Schwellwertschalter auf einem Interrupt-Eingang:] Hardware übernimmt die Signalverarbeitung mit Verstärkern und Frequenzfiltern. Anschließend schalten die Signalspitzen einen Transistor, der als Schwellwertschalter dient. Das binäre Signal kann ein Interrupt auf einem Mikrocontroller auslösen, worauf dieser sofort reagiert. Auch mit leistungsschwachen Controllern können dadurch exakte Laufzeitmessungen durchgeführt werden. Die Signalverarbeitung durch die Hardware ist entscheidend für die Reichweite und Störempfindlichkeit der Messungen.
\end{description}

In dieser Arbeit kommt ein Schwellwertschalter zum Einsatz. Hardware filtert und verstärkt das Signal, damit ein sauberes Durchschalten auch bei geringen Signalamplituden möglich ist. Das Signal wird weder gleichgerichtet, noch geglättet, sodass der Mikrocontroller die Perioden durch einzelne Pulse detektieren kann. Das bietet die Möglichkeit mit geringem Aufwand Störungen zu erkennen, z.B. kann ein einzelner Puls ignoriert werden. Außerdem ist eine einfache Frequenzauswertung auch möglich, wie z.B. Geschwindigkeitsmessungen über den Dopplereffekt, indem man die Pulse über eine bestimmte Zeit zählt.\\


