%!TEX root = ../dokumentation.tex

\chapter{Platinen und Hardware}
Die entworfenen Schaltungen wurden zuerst einzeln auf dem Steckbrett aufgebaut und getestet. Nachdem .. wurden sie gemeinsam auf eine Lochrasterplatine gelötet. Durch sauberes Planen und genaues Arbeiten sollten die Schaltungen genauso funktionieren, wie bei den Tests bestätigt wurde. Jedoch lassen sich Seiteneffekte und Wechselwirkungen zwischen den Schaltungen nicht ausschließen.


\section{Prototyp-Platine}
Die Prototyp-Platine war der erste Versuch ein voll-funktionsfähiges System aus den einzelnen Komponenten aufzubauen. Mit ihr sollten die Funktionen getestet, die Firmware für die Mikrocontroller entwickelt, sowie die Abstandsmessung in Reichweite, Genauigkeit und Zuverlässigkeit bewertet werden.\\
Die Platine beinhaltet zudem eine 7-Segment-Anzeige mit vier Ziffern, über die Messergebnisse ausgegeben werden können.

%TODO Bild von Platine mit Anzeige

\subsection{Aufbau}

%TODO Bild von 2D-Ansicht aus Platineneditor, Bereiche farbig markiert

\subsection{Steuerung, Funktionsweise}


\subsection{Probleme}



\section{Sensorplatinen}
Die Sensorplatinen sind abgespeckte Versionen der Prototyp-Platine, wobei die Schaltungen durch die gesammelten Erkenntnisse verbessert und vereinfacht wurden. Anders als die Prototyp-Platine sind sie nicht mit einer Anzeige ausgestattet, sodass ein Stand-Alone Betrieb ohne Kommunikationsverbindung per \ac{TWI} nicht möglich ist. Es gibt zwei Sensorplatinen, die zusammen mit der Prototyp-Platine für die Positionsbestimmung eingesetzt werden. Die Pinbelegung des Verbindungssteckers ist identisch mit der vorherigen Platine.

%TODO Bild von Platine, möglicherweiße auch 3D Ansicht aus KiCAD


\subsection{Verstärkerschaltung}



\subsection{Empfängerschaltung}



\section{Verteiler-Platine und Verkabelung}

