%!TEX root = ../dokumentation.tex

% Kapitel als Übersicht des Projekts
% Inhalt: Zielsetzung, Entiwcklung/Umsetzung, Hardware, Architektur, Tools


\chapter{Planung und Durchführung}


\section{Tools}
Um Schaltungen zu entwerfen sind einige Programme zum Einsatz gekommen. Damit ließen sich die Funktionen überprüfen, Frequenzgänge berechnen und es konnten Platinen mit den fertigen Schaltplänen designt werden.
\begin{description}
	\item[KiCAD] ist ein open-source Programm, mit dem modulare Schaltpläne gezeichnet werden können, woraus sich anschließend CAD-Modelle einer Platine erstellen lassen. Das Programm wird maßgeblich am CERN entwickelt und unterstützt viele zusätzliche Funktionen. So kann z.B. eine Netzliste für die \ac{SPICE}-Engine erstellt werden, mit der die Schaltungen simuliert werden können.\footnote{\url{http://www.kicad-pcb.org/}}
	\item[LTspice] ist ein Tool von \textit{Linear Technology} um analoge und digitale Schaltungen zu simulieren, insbesondere mit den firmeneigenen Operationsverstärkern. Das Programm basiert auf dem Simulator \ac{SPICE} und besitzt einen eigenen Schaltungseditor.\footnote{\url{http://www.linear.com/designtools/software/\#LTspice}}
	\item[AktivFilter] ist ein Tool um Frequenzfilter mit Operationsverstärkern zu entwerfen. Aus einigen Vorgaben berechnet das Programm ein Bode-Diagramm, dimensioniert die Bauteile und stellt einen Schaltplan zur Verfügung.\footnote{\url{http://www.softwaredidaktik.de/filter/}}
\end{description}


