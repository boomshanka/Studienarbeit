%!TEX root = ../dokumentation.tex

% Kapitel über Schaltungen, Signalerzeugung und -verarbeitung
\chapter{Schaltungen}


% Section über Sender-Schaltung, Signalerzeugung
\section{Sender} %TODO Sender Einleitung
Der Ultraschallsender wird durch ein $40 kHz$-Signal angeregt und erzeugt einen Schallkegel mit etwa $80^\circ$ Abstrahlwinkel. Der Schallpegel sollte dabei möglichst hoch sein, um die erfolgreiche Signalauswertung des reflektierten Signals auch über weite Distanzen und geringen Reflexionsfaktoren zu ermöglichen. Es wird empfohlen, den Sender mit einem Rechtecksignal anzusteuern. Durch sein sehr schmales Spektrum/Frequenz (FIXME) schwingt er dennoch sinusförmig ohne Oberwellen zu erzeugen und nimmt dabei am meisten Energie auf. (FIXME bei gleichbleibender Amplitude im vgl zu anderen Signalen/Sinus). Weiter mit Kapazität des Senders


\subsection{Signalerzeugung}
Das Rechtecksignal wird von dem Mikrocontroller erzeugt. Er besitzt einen Hardwarezähler, der mit einem Ausgangspin verbunden werden kann. Damit lässt sich ein präzises Signal erzeugen, ohne die Rechenleistung des Controllers einzuschränken.\\
Der Sender hat eine maximale Spitze-Spitze-Spannung von $U_{SS} = 20V$, das ausgegebene Signal des Mikrocontrollers wechselt jedoch nur zwischen $0V$ und $5V$. Die Sendeleistung wäre zu schwach, außerdem sollte der Ausgang des Mikrocontrollers nicht direkt belastet werden, da er nur sehr begrenzt Strom liefern kann. Aus diesen Gründen wurden mehrere Sendeverstärker entwickelt und untersucht, ob sie geeignet sind.



\subsection{Erste Schaltung: Verstärkung mit Operationsverstärker}
\subsubsection{Idee} % FIXME Differenzverstärker
Mithilfe eines Operationsverstärkers lassen sich einfach Spannungen verstärken. Mit der Komparatorschaltung kann das $0/5V$-Signal auf $-10/+10V$ umgesetzt werden. Gewählt wurde der Operationsverstärker \textit{LM258}, da seine Spannungsversorgung sehr hoch gewählt werden kann und er eine Bandweite von $1.1MHz$ hat, was für diese Zwecke ausreichen sollte.
\begin{figure}[H]
\centering
\includegraphics[scale=0.5]{images/komparatorschaltung.jpg}
\caption{Einfache Komparatorschaltung zum Umsetzen der Spannungen} \label{img:I1} %TODO: LM258/358
\end{figure}

\subsubsection{Problematik}
Bereits bei der Planung der Schaltung wurde befürchtet, dass der maximale Strom des Operationsverstärkers für ein scharfes Signal sehr knapp dimensioniert ist ($I_{Source}=30mA; I_{Sink}=40mA$). Nimmt man an, der Operationsverstärker besitzt einen Ausgangswiderstand von $0 \Omega$ und die Ladespannung beträgt $20V$, ist die Zeit zum vollständigen Laden des Senders:
\begin{equation}
t=\frac{Q}{I}=\frac{C*U}{I}=\frac{2.55nF*20V}{30mA}=1.7\mu s
\end{equation}
Im Vergleich zur Pulsdauer von $12.5\mu s$ ist dieser Wert in Ordnung. Allerdings besitzt nur ein idealer Operationsverstärker einen $0\Omega$-Ausgangswiderstand und der Sender ist auch kein idealer Kondensator. Deshalb wurde erwartet, dass die Flanken etwas abgeflacht werden.\\
Tatsächlich wurde jedoch ein viel wichtigerer Wert im Datenblatt übersehen. Die \textit{Slew Rate} gibt an, wie schnell sich die Spannung des OPs ändern kann. Beim \textit{LM258} beträgt sie $0.6V/\mu s$. Das Ausgangssignal entsprach aus diesem Grund einem Dreiecksignal und schwankte nur um wenige Volt.
% Bild von Simulation
\begin{figure}[H] %TODO: Kontrast von Bild erhöhen. Außerdem stimmen die 20V/µs nur ungefähr.
\centering
\includegraphics[scale=0.6]{images/signalverlauf_opamps.png}
\caption{Simulierter Signalverlauf von zwei Operationsverstärkern. Rot: Ansteuersignal. Grün: Slew-Rate $0.6V/\mu s$. Blau: Slew-Rate $20V/\mu s$.} \label{img:I2}
\end{figure}

\subsubsection{Fazit}
Die Spannungsverstärkung mit dem Operationsverstärker \textit{LM258} hat sich als nicht praktikabel herausgestellt, da er zu langsam reagiert. Für diese Zwecke können jedoch spezielle \textit{Komparatoren} verwendet werden, die sich wie Operationsverstärker verhalten, nur viel schneller und ungenauer sind. Die Genauigkeit spielt beim Umschalten keine Rolle.\\
Nach dieser Feststellung wurden Schaltungen mit Transistoren untersucht, da sie höhere Leistungen und schnellere Reaktionszeiten zulassen. Die Lösung mit Komparator wurde für den Fall, dass die weiteren Schaltungen keine befriedigende Ergebnisse liefern, für später aufgehoben.


\subsection{Zweite Schaltung: Verstärkung mit Transistor}
Diese Schaltung sollte möglichst einfach umsetzbar sein, da sie auch für Testzwecke verwendet werden soll. Die Spannungen sollten mit Bipolartransistoren geschaltet werden.

\subsubsection{Emitterschaltung}


\subsubsection{Spannungsbereich ausschöpfen}
Als Spannungsversorgung stehen $+5V$, $+12V$ und $-12V$ zur Verfügung. Es gibt verschiedene Schaltungen, die unterschiedliche Spannungsbereiche bieten:
\begin{description}
	\item[NPN-Transistor:] Gebräuchliche Schaltung um einen Schalter mit einen Transistor zu realisieren. Die $5V$ werden auf $12V$ verstärkt.
	\item[PNP-Transistor:] Mit nur einem Transistor kann das Signal auf $-12V$-$+5V$ umgesetzt werden. Die Spitze-Spitze-Spannung beträgt $17V$.
	\item[Zwei Transistoren:] Kombiniert man beide Verstärker, kann das Signal auf $-12V$-$+12V$ verstärkt werden. Da die Spitze-Spitze-Spannung des Senders um $4V$ überstiegen wird, muss sie mit einer Z-Diode begrenzt werden. Sie wird in Reihe entgegen der Spannung geschaltet, sodass an ihr die nötige Spannung abfällt.
	\item[Wechselsignal:] Der Sender ist nicht gegen Masse geschaltet, sondern erhält am anderen Pin das invertierte Signal. Dadurch lässt sich die Spitze-Spitze-Spannung verdoppeln. Es wird mindestens ein Transistor mehr benötigt, um die zweite Seite zu schalten. Allerdings bietet sich dafür keine Spannung an, da die $12V$ zu groß und die $5V$ zu klein sind, um die maximalen $20V$ zu erreichen.
\end{description}

%TODO: Bild

\subsubsection{Nachteile}



\subsection{Dritte Schaltung: Gegentaktausgangsstufe, Push/Pull}

\subsubsection{Idee}


% Section über Empfänger und Signalauswertung
\section{Empfänger}
Der Empfänger ist einer der wichtigsten Bausteine bei der Abstandsmessung mit Ultraschall. Nur durch eine zuverlässige Signalverarbeitung kann der Zeitpunkt exakt bestimmt werden, an dem das Signal eingetroffen ist. Im Prinzip gibt es zwei Möglichkeiten die Signalauswertung durchzuführen:
\begin{description} % FIXME: Vielleicht in eigenen sections sinnvoll
	\item[Signalauswertung mit AD-Wandler:] Das empfangene Signal wird vorverstärkt und ständig von einem AD-Wandler ausgelesen. Die Auswertung kann über einfache Schwellwerte bis hin zu komplexen Filterfunktionen, z.B. mithilfe einer Fourier-Transformation, beliebig aufwendig gestaltet werden. Eine Signalvorverarbeitung ist auch denkbar, z.B. Gleichrichten und Glätten, wenn nur die Auslenkung betrachtet wird. Software kann sich auf die Umgebungsbedingungen anpassen, ist jedoch ungenauer und langsamer als die entsprechende Hardwarelösung. Eine genaue Frequenzauswertung erfordert eine hohe Abtastrate und komplexe Software.
	\item[Schwellwertschalter auf einem Interrupt-Eingang:] Hardware übernimmt die Signalverarbeitung mit Verstärkern und Frequenzfiltern. Anschließend schalten die Signalspitzen einen Transistor, der als Schwellwertschalter dient. Das binäre Signal kann ein Interrupt auf einem Mikrocontroller auslösen, worauf dieser sofort reagiert. Auch mit leistungsschwachen Controllern können dadurch exakte Laufzeitmessungen durchgeführt werden. Die Signalverarbeitung durch die Hardware ist entscheidend für die Reichweite und Störempfindlichkeit der Messungen.
\end{description}

In dieser Arbeit kommt ein Schwellwertschalter zum Einsatz. Hardware filtert und verstärkt das Signal, damit ein sauberes Durchschalten auch bei geringen Signalamplituden möglich ist. Das Signal wird weder gleichgerichtet, noch geglättet, sodass der Mikrocontroller die Perioden durch einzelne Pulse detektieren kann. Das bietet die Möglichkeit mit geringem Aufwand Störungen zu erkennen, z.B. kann ein einzelner Puls ignoriert werden. Außerdem ist eine einfache Frequenzauswertung auch möglich, wie z.B. Geschwindigkeitsmessungen über den Dopplereffekt, indem man die Pulse über eine bestimmte Zeit zählt.\\


\subsection{Titel}



