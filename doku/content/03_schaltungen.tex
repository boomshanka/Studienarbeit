%!TEX root = ../dokumentation.tex

% Kapitel über Schaltungen, Signalerzeugung und -verarbeitung
\chapter{Schaltungen}
Für jede Aufgabe wurden verschiedene Schaltungen entwickelt und auf ihre Funktionalität überprüft. Probleme konnten bei der Weiterentwicklung vermieden werden, sodass die resultierende Schaltungen optimale Ergebnisse erzielen.

% Section über Sender-Schaltung, Signalerzeugung
\section{Sender}
Der Ultraschallsender wird durch ein $40 kHz$-Signal angeregt und erzeugt einen Schallkegel mit etwa $80^\circ$ Abstrahlwinkel. Der Schallpegel sollte möglichst hoch sein, um die erfolgreiche Signalauswertung des reflektierten Signals auch über weite Distanzen und geringen Reflexionsfaktoren zu ermöglichen. Bei voller Ausnutzung der Ansteuerspannung können Schallpegel bis zu $120dB$ erreicht werden, das entspricht in etwa der Lautstärke eines startenden Düsenflugzeugs. Allerdings lässt der Schalldruck mit steigendem Abstand schnell nach und ist für das menschliche Gehör weder wahrnehmbar, noch schädlich. Der hohe Abstrahlwinkel hat auch zur Folge, dass nur ein Bruchteil der Schallwelle am Objekt auftrifft und reflektiert werden kann.\\
Im Datenblatt wird empfohlen, den Sender mit einem Rechtecksignal anzusteuern. Durch sein schmales Frequenzband schwingt er dennoch Sinus-förmig. Der Sender besteht aus einem Piezoelement, dessen vereinfachtes Ersatzschaltbild ein Kondensator ist. Laut Datenblatt besitzt er eine Kapazität von $2.55nF$.\\


\subsection{Signalerzeugung}
Das Rechtecksignal wird von dem Mikrocontroller erzeugt. Er besitzt einen Hardwarezähler, der mit einem Ausgangspin verbunden werden kann. Damit lässt sich ein präzises Signal erzeugen, ohne die Rechenleistung des Controllers einzuschränken.\\
Der Sender hat eine maximale Spitze-Spitze-Spannung von $U_{SS} = 20V$, das ausgegebene Signal des Mikrocontrollers wechselt jedoch nur zwischen $0V$ und $5V$. Die Sendeleistung wäre zu schwach, außerdem sollte der Ausgang des Mikrocontrollers nicht direkt belastet werden, da er nur sehr begrenzt Strom liefern kann. Aus diesen Gründen wurden mehrere Sendeverstärker entwickelt. Dafür stand eine Spannungsversorgung mit den Spannungen $+5V$, $+12V$ und $-12V$ zu Verfügung.



\subsection{Erste Schaltung: Verstärkung mit Operationsverstärker}
\subsubsection{Idee} % FIXME Differenzverstärker
Mithilfe eines Operationsverstärkers lassen sich einfach Spannungen verstärken. Mit einer Komparatorschaltung kann das Steuersignal auf $-10V..+10V$ umgesetzt werden. Der Operationsverstärker \textit{LM258} ist zum Einsatz gekommen, da seine Spannungsversorgung sehr hoch und flexibel gewählt werden kann und er eine relativ hohe Bandweite von $1.1MHz$ aufweist. Außerdem ist die maximale Ausgangsspannung um zwei Volt nach Oben und Unten eingeschränkt, sodass die $20V$ zum Ansteuern des Senders exakt erreicht werden.
\begin{figure}[H]
\centering
\includegraphics[scale=0.5]{images/komparatorschaltung.jpg}
\caption{Einfache Komparatorschaltung zum Umsetzen der Spannungen} \label{img:I1}
\end{figure}

\subsubsection{Problematik}
Bereits bei der Planung der Schaltung wurde befürchtet, dass der maximale Strom des Operationsverstärkers für ein scharfes Signal sehr knapp dimensioniert ist ($I_{Source}=30mA; I_{Sink}=40mA$). Nimmt man an, der Operationsverstärker besitzt einen Ausgangswiderstand von $0 \Omega$ und die Ladespannung beträgt $20V$, ist die Zeit zum vollständigen Laden des Senders:
\begin{equation}
t=\frac{Q}{I}=\frac{C*U}{I}=\frac{2.55nF*20V}{30mA}=1.7\mu s
\end{equation}
Im Vergleich zur Pulsdauer von $12.5\mu s$ ist dieser Wert in Ordnung. Allerdings besitzt nur ein idealer Operationsverstärker einen $0\Omega$-Ausgangswiderstand und der Sender ist auch kein idealer Kondensator ohne seriellen Widerstand. Deshalb wurde erwartet, dass die Flanken etwas abgeflacht werden.\\
Tatsächlich wurde jedoch ein viel wichtigerer Wert im Datenblatt übersehen: Die \textit{Slew Rate} gibt an, wie schnell sich die Spannung des OPs ändern kann. Beim \textit{LM258} beträgt sie $0.6V/\mu s$. Das Ausgangssignal entsprach aus diesem Grund einem Dreiecksignal und schwankte nur um wenige Volt.
% Bild von Simulation
\begin{figure}[H]
\centering
\includegraphics[scale=0.6]{images/signalverlauf_opamps.png}
\caption{Simulierter Signalverlauf von zwei Operationsverstärkern} \label{img:I2}
Rot: Ansteuersignal. Grün: Slew-Rate $0.6V/\mu s$. Blau: Slew-Rate $20V/\mu s$.
\end{figure}

\subsubsection{Fazit}
Die Spannungsverstärkung mit dem Operationsverstärker \textit{LM258} hat sich als nicht praktikabel herausgestellt, da er zu langsam reagiert. Für diese Zwecke können jedoch spezielle \textit{Komparatoren} verwendet werden, die sich wie Operationsverstärker verhalten, jedoch viel schneller und ungenauer sind. Die Genauigkeit spielt beim Umschalten keine Rolle.\\
Nach dieser Feststellung wurden Schaltungen mit Transistoren untersucht, da sie höhere Leistungen und schnellere Reaktionszeiten zulassen.


\subsection{Zweite Schaltung: Verstärkung mit Transistor} \label{schaltung:transistor}
%TODO Sollen Berechnungen von Strömen und Widerständen hier noch rein?
Der Bipolartransistor ist ein stromgesteuertes Halbleiterbauelement, der als Verstärker und Schalter eingesetzt wird. Die Emitterschaltung eignet sich für hohe Spannungs- und Stromverstärkungen. Im übersteuerten Zustand lassen sich damit ohmsche Lasten elektronisch schalten. Gemeinsamer Bezugspunkt ist der Emitter - daher der Name der Schaltung. Der Strom von Basis zu Emitter steuert den maximalen Stromfluss der Schaltstrecke von Kollektor zu Emitter. Um ein digitales Signal zu verstärken, wird es über den Basiswiderstand auf den Transistor gegeben und ein weiterer Widerstand vor den Kollektor zur resultierenden Spannung geschaltet. Leitet der Transistor, fällt die Spannung am Widerstand ab. Sperrt er, fällt an ihm die Spannung ab und der Widerstand zieht das Potential vor dem Transistor auf den gewünschten Wert. Das resultierende Signal ist um $180^\circ$ phasenverdreht, d.h. es wird invertiert aber nicht verzögert. Der Transistor wird im übersteuerten Bereich betrieben, damit er sauber durchschalten kann und keine Verluste durch Spannungsabfall an ihm entstehen. \\


\subsubsection{Spannungsbereich ausschöpfen}
Als Spannungsversorgung stehen $+5V$, $+12V$ und $-12V$ zur Verfügung. Es gibt verschiedene Schaltungen, die unterschiedliche Spannungsbereiche bieten:
\begin{description}
	\item[NPN-Transistor:]\hspace{1pt}\\
		Gebräuchliche Schaltung um einen Schalter mit einen Transistor zu realisieren. Die $5V$ werden auf $12V$ verstärkt.
	\item[PNP-Transistor:]\hspace{1pt}\\
		Mit nur einem Transistor kann das Signal auf $-12$-$+5V$ umgesetzt werden. Die Spitze-Spitze-Spannung beträgt $17V$.
	\item[Zwei Transistoren:]\hspace{1pt}\\
		Kombiniert man beide Verstärker, kann das Signal auf $-12V$-$+12V$ verstärkt werden. Da die Spitze-Spitze-Spannung des Senders um $4V$ überstiegen wird, muss sie mit einer Z-Diode begrenzt werden. Sie wird in Reihe entgegen der Spannung geschaltet, sodass an ihr die nötige Spannung abfällt.
	\item[Wechselsignal:]\hspace{1pt}\\
		Der Sender ist nicht gegen Masse geschaltet, sondern erhält am anderen Pin das invertierte Signal. Dadurch lässt sich die Spitze-Spitze-Spannung verdoppeln. Es wird mindestens ein weiterer Transistor benötigt, um die zweite Seite zu beschalten. Dafür kann das Ausgangssignal des ersten Transistors genutzt werden, da es um $180^\circ$ phasenverdreht zum Eingangssignal ist. Der Vorteil dieser Schaltung ist, dass die Spannung halbiert werden kann, ohne Sendeleistung einzubüßen.
\end{description}

\begin{figure}[H]
\centering
\includegraphics[scale=0.6]{images/transistorschaltungen.png}
\caption{Verstärkungsschaltungen für $12V$, $17V$ und $20V$ Spannungsdifferenz} \label{img:I3}
\end{figure}

\subsubsection{Nachteile}
Die Emitterschaltung ist eine einfache Verstärkerschaltung und eignet sich hervorragend um ohmsche Lasten zu schalten. Der Sender ist jedoch eine kapazitive Last, die zu jeder Flanke geladen oder entladen werden muss. Über den Lastwiderstand wird die Kapazität geladen, wenn der Transistor sperrt. Solange er leitend ist, fließt ein konstanter Strom über diesen Widerstand, der für den Sender nicht benötigt wird. Bei der Dimensionierung der Lastwiderstände wird ein hoher Stromfluss in Kauf genommen, um die gewünschte Flankensteilheit, bedingt durch die RC-Ladekurve, zu erreichen. Statt einem Ein- und Ausschalten wäre ein Umschalten zwischen zwei Spannungen sinnvoller. Diesen Ansatz verfolgt die nächste entworfene Schaltung.

%TODO Bild überarbeiten
\begin{figure}[H]
\centering
\includegraphics[width=(\textwidth)]{oszi/15-04-01/2.png}
\caption{RC-Ladekurven hinter einem Transistor (Oben Ansteuersignal)} \label{img:I4}
\end{figure}

\subsection{Dritte Schaltung: Gegentaktstufe} \label{schaltung:gegentakt}
Anstatt die Spannung über den Lastwiderstand am Kollektor eines Transistors auf den richtigen Wert zu ziehen, kann die Spannung über zwei Transistoren, die sich im Gegentakt öffnen und schließen, direkt erreicht werden. Die Vorteile dabei sind, dass der Sender über sehr kleine Widerstände schnell geladen werden kann und dennoch kein unnötiger Stromfluss entsteht. Das Prinzip der Gegentaktstufe wird in verschiedenen Bereichen angewendet: Von schnellen Logik-ICs über verschiedene Treiberbausteine bis hin zu Leistungsendstufen. Durch die Symmetrie und den geringen Eigenstromverbrauch lassen sich hohe Lasten sehr schnell An- und Abschalten. Allerdings ist eine Gegentaktstufe nicht linear, wodurch sie meist ungeeignet für analoge Anwendungen ist. In diesem Fall wird nur ein digitales Umschalten benötigt.



\subsubsection{Gleichzeitiges Öffnen der Transistoren}
Ein hohes Risiko der Schaltung ist, dass ein Kurzschluss entsteht, sobald beide Transistoren gleichzeitig geöffnet sind. Die hohen Ströme würden nicht zuletzt den thermischen Tod der Transistoren bedeuten. Die Schaltung muss so gestaltet werden, dass ein gleichzeitiges Öffnen nicht möglich ist. Durch folgenden Aufbau kann dies erreicht werden:\\
Die Emitter zweier Transistoren, jeweils einer vom Typ NPN und PNP, werden zusammengelegt. An dieser Stelle wird auch der Sender angeschlossen. Ein NPN-Transistor öffnet nur, wenn ein Strom von Basis zu Emitter fließt. Für den PNP-Transistor muss er in die andere Richtung fließen. Werden nun auch die Basispins dieser beiden Transistoren verbunden, kann nur einer der zwei Fälle auftreten - ein gleichzeitiges Öffnen ist deshalb nicht mehr möglich.
%TODO Nummern ins Bild und im Text verlinken
\begin{figure}[H]
\centering
\includegraphics[scale=0.6]{images/push_pull.png}
\caption{Gegentaktstufe als Verstärkerschaltungen} \label{img:I5}
\end{figure}


\subsubsection{Verzögerte Reaktionen der Transistoren}
Dioden haben an den Übergängen der Dotierungen kapazitive Eigenschaften. Besonders bei hohen Spannungen wirken sie sich sehr negativ auf die Schaltgeschwindigkeit aus, da die Ladung erst abtransportiert werden muss, bevor sich der Zustand des Transistors ändert. Das führt dazu, dass sie zu spät schließen und die Signale verformt werden. Über mehrere Transistoren summiert sich dieser Effekt auf. Anfangs wurde versucht, die Steuerspannung mithilfe von in Reihe geschalteten \textit{Zener-Dioden} zu begrenzen. Auch eine Diode von Basis zu Kollektor (NPN), bzw. von Kollektor zur Basis (PNP) verringert die Schaltgeschwindigkeit. Überschüssige Ladungsträger fließen über die niederohmige Kollektor-Emitter-Strecke ab, wodurch ein Übersättigen des Transistors vermieden wird. Bei der Gegentaktstufe tritt jedoch der Fall ein, bei dem der Sender vollständig geladen ist und kein Strom mehr fließt. Ohne Stromfluss fällt an den Z-Dioden keine Spannung ab und die Transistoren übersättigen erneut.\\ %TODO Warum schlechtes Wort?!
Eine akzeptable Lösung wurde gefunden, indem die Arbeitspunkte der Transistoren über Widerstandsverhältnisse (R13 und R14 sowie R15 und R16, siehe Abbildung \ref{img:I5}) angepasst wurden. Mithilfe von Schaltungssimulationen konnten die Verhältnisse genau bestimmt werden, bei denen die Transistoren noch sauber schalten und gleichzeitig vor dem Schließen kaum Ladungsträger abtransportiert werden müssen. Die Simulationen waren so genau, dass mit nur geringfügigen Änderungen der Widerstandswerte auch in der Realität optimale Ergebnisse erzielt wurden.

\subsubsection{Aufbau und Funktionsweise}
Die Verstärkerschaltung setzt das Ansteuersignal von $0..5V$ auf $-8..+12V$ um, sodass der Ultraschallsender mit $20V$ Spannungsdifferenz angesteuert wird. Die vier Volt Differenz von der Spannungsquelle mit $-12V$ werden mithilfe einer Zener-Diode erreicht, damit der Sender nicht durch Überspannung beschädigt wird. Die Umsetzung wird mit drei Schritten erreicht: In den ersten zwei Stufen (mit T4 und T3, siehe Abbildung \ref{img:I5}) wird das Signal auf $-12..+12V$ verstärkt (Prinzip der dritten Schaltung aus Abbildung \ref{img:I3}), die letzte Stufe ist die eigentliche Gegentakt-Ausgangsstufe, wo das Umschalten der beiden Ausgangsspannungen stattfindet.
\begin{figure}[H]
\centering
\includegraphics[width=1\textwidth]{oszi/15-04-30/1.png}
\caption{Ausgangssignal der Gegentaktstufe mit angeschlossenen Sender} \label{img:I8}
\end{figure}

\subsubsection{Fazit}
Die Gegentakt-Ausgangsstufe ist im Vergleich zu den bisherigen Schaltungen sehr aufwändig. Mit ihr lassen sich sehr große Lasten schalten, gleichzeitig ist sie sehr stromsparend. Sie ist leistungsmäßig für diesen Zweck jedoch überdimensioniert, die leistungsschwächere Ausgangsstufe von einem oder mehreren Logik-ICs hätten diesen Zweck auch erfüllen können.


\subsection{Oberwellen und Schwingungen} %TODO Zeit!
Der Grund, der zur Entwicklung von leistungsstärkeren Verstärkern führte, waren die unbefriedigenden Signalverläufe am Ausgang der Sendeverstärkers. Die Pegel wurden nur teilweise erreicht und nicht gehalten. Der Fehler war, dass dafür die falschen Ursachen angenommen wurden. Schwingungen durch Oberwellen aus den steilen Flanken wurden zu Beginn aus mehreren Gründen ausgeschlossen:
\begin{itemize}
	\item Das Datenblatt empfiehlt die Ansteuerung der Sender mit Rechtecksignalen
	\item Das Frequenzband des Senders ist sehr schmal, andere Frequenzen werden stark gedämpft
	\item Die Flanke beim Schließen der Transistoren sind steil, beim Öffnen (Ladestrom fließt über Widerstand) gibt es RC-Ladekurven-ähnliche Verformungen der Flanken
	\item Der Signalverlauf am Empfänger war wie erwartet ein sauberes Sinus-Signal
\end{itemize}
Es wurde vermutet, dass die Ströme nicht ausreichend sind, um die Kapazitäten in den geringen Pulszeiten zu versorgen. Allerdings erbrachte der Gegentaktverstärker trotz den hohen Strömen auch nicht das erhoffte Ergebnis. Wieder wurden die Pegel nicht sauber gehalten, nur die Flankensteilheit wurde verbessert.\\
Ein Hinweis auf das Problem gab es jedoch: Die Ausgangsspannung schwingt auch weiter, nachdem das Steuersignal keine Pulse mehr sendet. Dieses Verhalten tritt nicht auf, wenn der Ultraschallsender nicht angeschlossen ist. Tatsächlich stammen die Schwingungen vom Sender. In ihm wird die elektrische Schwingung in eine mechanische umgesetzt. Das schwingende Element, dass die Schallwellen erzeugt, besitzt eine Trägheit, die es mechanisch nachschwingen lässt und dadurch auch elektrische Signale rückkoppelt.\\ %TODO: Rückkoppeln? Besseres Wort?
Die unsauberen Pegel innerhalb der Pulse deuten darauf hin, dass es auch zu Schwingungen mit höheren Frequenzen kommt. Sie sind in den steilen Flanken enthalten und regen das mechanische Piezoelement an. Dabei kann es sich auch um harmonische Oberschwingungen handeln, d.h. Schwingungen mit der n-Fachen Frequenz der Resonanzschwingung von $40 kHz$, die weniger stark gedämpft werden.
\begin{figure}[H]
\centering
\includegraphics[width=(\textwidth), angle=0]{oszi/15-04-01/4.png}
\caption{Oberwellen und Schwingungen der Ausgangsstufe (oben)} \label{img:I7}
\end{figure}

\subsubsection{Lösung}
Um Oberwellen zu vermeiden, können die Sender statt mit Rechtecksignalen mit einem Sinus angesteuert werden. Die meisten hochfrequenten Anteile lassen sich auch schon mit einem Tiefpass entfernen, der die Flanken etwas abflacht. Der Versuch die Verstärker immer leistungsfähiger zu machen, führte vielmehr dazu, dass noch mehr hochfrequente Anteile entstehen konnten. Es wurde versucht die Kapazitäten zu bekämpfen, die jedoch das echte Problem nicht darstellten.\\
Da im Datenblatt die Sender mit Rechtecksignalen angesteuert werden und die empfangenen Signale einem sauberen Sinus entsprechen, können die Oberwellen auch ignoriert werden. Der Grund hierfür ist, dass der Empfänger wiederum einen schmalen Frequenzbereich aufweist und das Signal filtert. Zudem ist es möglich, dass die Oberschwingungen nur auf der elektrischen Seite entstehen und die mechanischen Wellen davon unbeeinflusst bleiben. Ein Leistungsverlust der Nutzfrequenz durch die störenden Anteile kann jedoch nicht ausgeschlossen werden.\\
Das Nachschwingen kann allerdings zu einem Problem werden, wenn die Signale den empfindlichen Empfänger beeinflussen, entweder durch mechanische oder elektrische Rückkopplung. Die mechanischen Schwingungen können reduziert werden, indem zwischen Sender und Empfänger eine größere Distanz eingehalten wird und durch Schaumstoff zwischen ihnen mögliche akustische Wellen gedämpft werden. Elektrische Störungen werden mit Abblock-Kondensatoren geschwächt. Sie befinden sich zwischen den Versorgungsspannungen und der Masse und schließen hochfrequente Anteile kurz. Ansonsten muss eine Totzeit berücksichtigt werden, bis der Empfänger das Nutzsignal ohne Störungen empfangen kann. Im Datenblatt wird eine minimale Distanz der Messungen von $20cm$ angegeben, wahrscheinlich genau aus den genannten Gründen.
\begin{figure}[H]
\centering
\includegraphics[width=(\textwidth), angle=0]{oszi/15-04-23/2_01.png}
\caption{Schwingungen der Ausgangsstufe (unten) und Rückkopplung am Empfänger} \label{img:I6}
\end{figure}


% Section über Empfänger und Signalauswertung
\section{Empfänger}
Der Empfänger ist einer der wichtigsten Bausteine bei der Abstandsmessung mit Ultraschall. Nur durch eine zuverlässige Signalverarbeitung kann der Zeitpunkt exakt bestimmt werden, an dem das Signal eingetroffen ist. Im Prinzip gibt es zwei Möglichkeiten, die Signalauswertung durchzuführen:
\begin{description} % FIXME: Vielleicht in eigenen sections sinnvoll
	\item[Signalauswertung per Software mit AD-Wandler:]\hspace{1pt}\\
	Das empfangene Signal wird vorverstärkt und ständig von einem AD-Wandler ausgelesen. Die Auswertung kann über einfache Schwellwerte bis hin zu komplexen Filterfunktionen, z.B. mithilfe einer Fourier-Transformation, beliebig aufwendig gestaltet werden. Eine Signalvorverarbeitung ist auch denkbar, z.B. Gleichrichten und Glätten, wenn nur die Auslenkung betrachtet wird. Software kann sich auf die Umgebungsbedingungen anpassen, ist jedoch ungenauer und langsamer als die entsprechende Hardwarelösung. Eine genaue Frequenzauswertung erfordert eine hohe Abtastrate und komplexe Software, die dementsprechend auch Rechenleistung benötigt.
	\item[Schwellwertschalter auf einem Interrupt-Eingang:]\hspace{1pt}\\
	Hardware übernimmt die Signalverarbeitung mit Verstärkern und Frequenzfiltern. Anschließend schalten die Signalspitzen einen Transistor, der als Schwellwertschalter dient. Das binäre Signal kann ein Interrupt an einem Mikrocontroller auslösen, worauf dieser sofort reagiert. Auch mit leistungsschwachen Controllern können dadurch exakte Laufzeitmessungen durchgeführt werden. Die Signalverarbeitung durch die Hardware ist entscheidend für die Reichweite und Störunempfindlichkeit der Messungen.
\end{description}
In diesem Projekt wird ein Transistor als Schwellwertschalter verwendet. Für die Signalvorverarbeitung wurden verschiedene Ansätze entwickelt und untersucht. Diese unterscheiden sich in Empfindlichkeit und Komplexität und wurden für die Anwendungsfälle entsprechend bewertet.



\subsection{Erster Entwurf: Frequenzfilterung und Pulsverstärkung} \label{schaltung:pulse}

\subsubsection{Idee}
In diesem Ansatz wird das Eingangssignal so verstärkt werden, damit jede Periode einen binären Puls erzeugt, der jeweils vom Mikrocontroller detektiert werden kann. Der Vorteil dieser Methode ist, dass auch einfache Frequenzanalysen möglich sind. So kann z.B. durch das Zählen dieser Pulse über den Dopplereffekt eine Geschwindigkeit bestimmt werden. Ein einzelner Puls kann auch als Störung erkannt und ignoriert werden.\\
\begin{figure}[H]
\centering
\includegraphics[width=(\textwidth), angle=0]{oszi/15-04-23/4_03.png}
\caption{Pulsverstärkung bei einer Laufzeitmessung} \label{img:I11}
Oben: Eingangssignal am Empfänger; Unten: Ausgangs-Pulse der Empfängerschaltung. Zu Beginn der Messung werden die Signale bereits vom Sender Rückgekoppelt, die Pulse auf der Rechten Seite stammen von der Reflektion.
\end{figure}
Um eine möglichst hohe Reichweite zu erzielen, sollte die Verstärkung der Signale sehr hoch gewählt werden. Allerdings werden damit auch Störsignale verstärkt, die fälschlicherweise als Ultraschallsignale interpretiert werden können. Um das zu vermeiden, wird das Eingangssignal mithilfe eines Bandpasses gefiltert.


\subsubsection{Umsetzung}
Der Bandpass wurde mithilfe eines aktiven Tiefpasses 2. Ordnung und zwei passive CR-Hochpässen realisiert, woraus ein Bandpassfilter 2. Ordnung resultiert. Der aktive Filter verstärkt das Signal zusätzlich. Auch die Signale oberhalb der Grenzfrequenz erfahren eine Dämpfung, die dadurch ausgeglichen werden kann. Das Signal wird vor und nach dem Tiefpass mit Operationsverstärkern verstärkt, anschließend wird eine Offsetspannung hinzu addiert. Das resultierende Signal wird auf die Basis eines Transistors gegeben, der bei kleinsten Impulsen bereits durchschaltet und einen binären Puls erzeugt. Verstärkungen und Offsetspannung können über Potentiometer exakt eingestellt werden.
\begin{figure}[H]
\centering
\includegraphics[width=(\textwidth), angle=0]{sim/verstaerker_offset.jpg}
\caption{Simulation der Pulsverstärkung} \label{img:Sim1}
Rot: Eingangssignal; Blau: Verstärktes Signal mit Offset; Grün: Binäre Pulse
\end{figure}
Der aktive Tiefpass wurde als \textit{Sallen-Key-Tiefpass} 2. Ordnung mit einem Operationsverstärker realisiert. Seine Verstärkung beträgt $3dB$. 
\begin{figure}[H]
\centering
%\includegraphics[width=(\textwidth), angle=0]{sim/}
\caption{Bode-Diagramm des aktiven Frequenzfilters} \label{img:bode}
\end{figure}
%TODO: Diagramm einfügen

\subsubsection{Probleme}
Die Operationsverstärker können eine Verstärkung von maximal 100 erreichen, bei höheren Werten bleibt der Ausgangspegel am Maximum oder Minimum hängen. Durch die serielle Verschaltung mehrerer Verstärker können dennoch höhere Faktoren erreicht werden. Allerdings besteht die Gefahr, dass dadurch mehr Störsignale eingekoppelt werden. Ein weiteres Problem der Operationsverstärker ist die geringe Slew-Rate. Ein periodisches Signal kann nur auf einen gewissen Pegel verstärkt werden, da das Signal ansonsten zu steil werden würde. Wenn das geschieht, wird die Signalform abgeflacht und verfälscht, wodurch auch andere Frequenzanteile entstehen, die von den Filtern wieder entfernt werden. Die maximale Verstärkung ist deshalb sehr eingeschränkt, jedoch sollte sie möglichst hoch sein, um den Transistor bei jedem Puls sauber durchzuschalten.\\
Der Verstärkungsfaktor bei aktiven Filtern muss kleiner als drei sein, ansonsten fängt der Filter an, in seiner Resonanzfrequenz zu schwingen. Auch kleinere Faktoren können bereits ein längeres Nachschwingen bewirken, das unter Umständen weitere Messungen verfälscht.
\begin{figure}[H]
\centering
\includegraphics[width=(\textwidth), angle=0]{sim/schwingen_bei_verstaerkung.png}
\caption{Signal eines schwingenden aktiven Frequenzfilters} \label{img:Sim2}
Verstärkungsfaktor: $1.7$; Rot: Ansteuersignal; Grün: Verstärktes Signal; Blau: Schwingender Ausgang des Filters
\end{figure}
Bei allen durchgeführten Experimenten wurden keinerlei Störsignale identifiziert, die vom Ultraschall-Empfänger aufgenommen und durch die Filter beseitigt wurden. Die Filter erhöhten die Komplexität der Schaltung und schränkten die Möglichkeiten ein, das Signal zu verarbeiten. Aus diesem Grund wurde entschieden, die Frequenzfilterung in den nachfolgenden Schaltungen nicht zu verwenden.
\begin{figure}[H]
\centering
\includegraphics[width=(\textwidth), angle=0]{oszi/15-04-23/3_01.png}
\caption{Signal vor (unten) und nach der Filterung} \label{img:I10}
\end{figure}

\subsubsection{Fazit}
In bestimmten Anwendungsfällen macht es Sinn, Perioden als einzelne Pulse zu verstärken. Für die Abstandsmessung ist diese Methode jedoch ungünstig, da sie exaktere Schaltungen benötigt und die Empfindlichkeit - und damit die Reichweite - eingeschränkt wird. Auch die verwendeten Operationsverstärker sind durch die geringe Slew-Rate ein Hindernis. Störsignale hingegen spielen keine Rolle, die aktiven Frequenzfilter können deshalb eingespart werden. Dadurch lassen sich auch potentielle Fehlerquellen und Seiteneffekte vermeiden.\\
Trotz den genannten Problemen, lassen sich mit dieser Schaltung sehr exakte Messungen in einer Spanne von $15cm$-$200cm$ durchführen. Die relative Genauigkeit befindet sich im Millimeter-Bereich, der absoluter Messfehler bewegt sich im Bereich von etwa $2cm$, abhängig von dem zu messenden Objekt.



\subsection{Zweiter Entwurf: Schaltung mit Integrierverstärker} \label{schaltung:integrator}

\subsubsection{Änderungen}
Einige Erkenntnisse aus der ersten Schaltung führten zu der neuen Lösung. Die Anforderungen wurden geändert und auf den Zweck der Abstandsmessung optimiert. Das Signal wurde nach einem ersten Vorverstärker gleichgerichtet und mit einem Tiefpass geglättet, sodass die Probleme der hohen Frequenzen wegfielen. Eine Frequenzfilterung mit einem aktiven Bandpass findet nicht mehr statt. Die Komplexität der Schaltung wurde deutlich minimiert, statt vier Operationsverstärkern kommen nur noch zwei zum Einsatz und die Empfindlichkeit wird über ein einziges Potentiometer eingestellt, statt über den ursprünglichen drei.

\begin{landscape}
	\begin{figure}
		\centering
		\includegraphics[width=(1.5\textwidth)]{images/empfanger2.png}
		\caption{Schaltplan der verbesserten Empfängerschaltung} \label{img:Empf2}
	\end{figure}
\end{landscape}

\subsubsection{Funktionsweise}
Ein empfangener Impuls wird über den Kondensator \textit{C1} aufsummiert. Über den parallelen Widerstand \textit{R8} fließt ein geringer Strom wieder ab, sodass sich der Kondensator langsam entlädt. Werden mehrere Pulse empfangen, steigt die Spannung soweit an, dass der darauf folgende Verstärker eine hohe Ausgangsspannung erzeugt. Diese wird über den Koppelkondensator \textit{C2} und dem Potentiometer, das die Offsetspannung für die Transistoren vorgibt, auf eine \textit{Darlington-Stufe} geleitet, die dank der doppelten Verstärkung von zwei Transistoren auch auf geringe Steuerströme reagiert. Sie wurde gewählt, um die Schaltung noch empfindlicher zu machen. Durch die integrierende Wirkung des Tiefpasses können kleine Impulse zu hohen Spannungen aufsummiert werden, sodass auch geringe Signalstärken verarbeitet werden können. Allerdings kann es dabei zu Verzögerungen kommen, da ein gewisser Spannungspegel erreicht werden muss und sich dieser Vorgang über mehrere Perioden hinziehen kann.
\begin{figure}[h]
	\centering
	\includegraphics[width=(\textwidth), angle=0]{oszi/15-05-15/3.png}
	\caption{Signalverlauf am Empfänger und am Integrierverstärker} \label{img:I12}
\end{figure}


\subsubsection{Fazit}
Die Schaltung erzielt ähnlich gute Ergebnisse, wie der erste Entwurf, benötigt dafür jedoch viel weniger Bauelemente. Die Komplexität konnte gesenkt werden und die Störempfindlichkeit durch die geringeren Frequenzen und Verstärkungsfaktoren, die nicht mehr so hoch gewählt werden müssen, konnte verbessert werden.

