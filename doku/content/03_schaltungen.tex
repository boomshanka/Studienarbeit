%!TEX root = ../dokumentation.tex

% Kapitel über Entwicklung der Schaltungen
% Tools, Aufgaben, Ideen, Planung, Vorgehensweise..

\chapter{Entwicklung der Schaltungen}



\section{Sender}
Der Ultraschallsender ist ein Lautsprecher, der mit einer Frequenz von $40kHz$ angeregt werden kann. Er hat einen sehr schmalen Frequenzgang (FIXME), weshalb er auch mit einem gepulsten Signal angeregt werden kann, ohne Oberwellen zu erzeugen. Die einfachste Möglichkeit zum Erzeugen eines $40kHz$-Signals ist mit einem \ac{PWM}-Singal, erzeugt mithilfe des Timers MIKROCONTROLLER BLA


\section{Empfänger}


\section{Tools}
Um Schaltungen zu entwerfen sind einige Programme zum Einsatz gekommen. Damit ließen sich die Funktionen überprüfen, Frequenzgänge berechnen und es konnten Platinen mit den fertigen Schaltplänen designt werden.
\begin{description}
	\item[KiCAD] ist ein open-source Programm, mit dem modulare Schaltpläne gezeichnet werden können, woraus sich anschließend CAD-Modelle einer Platine erstellen lassen. Das Programm wird maßgeblich am CERN entwickelt und unterstützt viele zusätzliche Funktionen. So kann z.B. eine Netzliste für die \ac{SPICE}-Engine erstellt werden, mit der die Schaltungen simuliert werden können.\footnote{http://www.kicad-pcb.org/display/KICAD/}
	\item[LTspice] ist ein Tool von \textit{Linear Technology} um analoge und digitale Schaltungen zu simulieren, insbesondere mit den firmeneigenen Operationsverstärkern. Das Programm basiert auf dem Simulator \ac{SPICE} und besitzt einen eigenen Schaltungseditor.\footnote{http://www.linear.com/designtools/software/\#LTspice}
	\item[AktivFilter] ist ein Tool um Frequenzfilter mit Operationsverstärkern zu entwerfen. Aus einigen Vorgaben berechnet das Programm ein Bode-Diagramm, dimensioniert die Bauteile und stellt einen Schaltplan zur Verfügung.\footnote{http://www.softwaredidaktik.de/filter/}
\end{description}


\newpage

\chapter*{Beispiel}

Das hier ist mit einer Quelle belegt.\footnote{\cite[42]{baumgartner:2002}} Quelle wird im Verzeichnis gedruckt.\\
Hier habe ich eine Fußnote\footnote{Ich bin eine Fußnote} und das hier ist ein \ac{Akronym}.
